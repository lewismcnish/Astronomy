\documentclass[]{article}
\usepackage{graphicx}
\usepackage{amsmath}
\usepackage{amssymb}
\usepackage{amsfonts}
\usepackage{fancyhdr}
\usepackage[headheight=65pt,tmargin=150pt,headsep=95pt]{geometry}
\usepackage{ragged2e}
\usepackage{array}
\usepackage{tabularx}



\graphicspath{{./images/}}

\pagestyle{myheadings}
\markright{Title\hfill 2663452m\hfill date\hfill}

\title{\textbf{Binary black hole detections from LIGO-VIRGO runs 1 and 2}}
\author{2663452m (University of Glasgow)}
\date{12/04/2023}



\begin{document}
\maketitle

\begin{abstract}

\end{abstract}
\twocolumn
\newpage





\section*{Introduction and Background}
Gravitational waves as first predicted by Albert Einstein in 1915 in his paper on
special and general relativity, are ripples in the fabric of space-time due to the acceleration of large masses
and have been notoriously hard to detect. That was until
the LIGO Michelson interferometer in Hanford and Livingston was complete in 2015.
A Michelson interferometer is a device that uses the interference of two beams of
light to detect small changes in the path distance of the two beams. A diagram of one can
be seen in Figure 1.
\begin{figure}[h]
    \includegraphics[width=6cm]{images/michelson_interferometer.png}
    \caption{Diagram of a michelson interferometer as used in LIGO.$^1$}
    \label{fig:michelson}
    \end{figure}
By using a Michelson interfermoeter in the LIGO experiment the small changes in
distance that are required can be detected and measured. These distances can be on the
order of 10$^{-21}$m this distance is calle dthe strain of the wave and
is the amount of stretching over the original length and can be approximated as
\begin{equation}h \approx \frac{GM}{c^2d}\left(\frac{v}{c}\right)^2 \label{eq:strain}\end{equation}
where $G$ is the gravitational constant, $M$ is the mass of the source, $c$ is the speed of light,
d is the distance to the source and $v$ is the velocity of the system.

caused by the passing of gravitational waves moving at the speed of light
through the
interferometer arms (which results in a shift in the interference pattern of the light beams).
The first detection of a gravitational wave was on the 14th of
September 2015, just 100 years after the publication of Einstiens paper.
The first detection was of a binary black hole merger, these mergers commonly
release a large amount of energy in the form of gravitational waves. This happens
because as the two black holes accelerate towards each other they warp the
space-time around them, and as they approach the point of coalescence the amplitude
of these waves massively increases, thus allowing them to be detected over the Background
noise. The run-down after merging is extremely quick and thus leaving a distinct peak
at the time of coalescence. In this report the first 11 detections of gravitational waves
as a result of binary black hole mergers will be analysed and discussed. Starting with GW150914.

\section*{GW150914}
To be able to carry out the analysis of the gravitaional waves it was necessary to set
up our workspace to be able to use some provided function and packages written by the
LIGO collaboration. This was done by first installing the LIGO lalsuite package for Python 3.10
and then importing the packages into our notebook. The package contains a number of useful
functions that will be discussed in more detail later in this report.
For the first detection of gravitational waves GW150914 the data was provided
by the university through the Jupyter Hub. Once this data was loaded in
the first step was to plot the strain against the time and identify by eye the peak
of the gravitational wave. This plot is shown in Figure 2.
\begin{figure}
    \includegraphics[width=6cm]{images/Signal_gw150914.png}
    \caption{top: Plot of the strain against time for GW150914 bottom:
    limited to shorter time to resolve the peak more.}
    \label{fig:GW150914}
\end{figure}
\newpage
From Figure 2 it can be seen that the peak of the gravitaional wave occurs at
around 3.2 seconds.
Knowing this time we can use the SCIPY package to generate a spectrogram of strain and frequency and from this a color plot can be
created which visualizes the amount of energy in the gravitational wave at a given
frequency and time. This plot is shown in Figure 3.
\begin{figure}[h]
    \includegraphics[width=6cm]{images/spectrogram_gw150914.png}
    \caption{Spectrogram of GW150914, showing the 'chirp' track of the gravitational wave.}
    \label{fig:spectrogram}
\end{figure}
\newline
In this plot the point where the energy is high across multiple frequencies
correlates with the same time as the peak in the strain in Figure 2.

Now that we have visualised the actual data form the gravitational wave, it would now
be useful to be able to compare this to the theorectical prediction of the event.
This can be generated using the make template function as supplied by the LIGO
collaboration. This function takes in the masses of the two blackholes and
the time, frequency, distance and uncertainty on the data. The function then returns
a strain and time array that can be used to plot the theorectical predictions.
This produces an ideal signal as seen in Figure 4, this is easier to see the signal as,
it no longer has any noise.
\begin{figure}[h]
    \includegraphics[width=6cm]{images/ideal_signal_gw150914.png}
    \caption{Theoretical prediction for GW150914}
    \label{fig:ideal_signal}
\end{figure}
\newline
Later this template will be overlayed on the true data to determine the goodness of fit.
\section*{Method}




\section*{Results}



\section*{Analysis}

\section*{Conclusion}

\section*{References}
\end{document}
