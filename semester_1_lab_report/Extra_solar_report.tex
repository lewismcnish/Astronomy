

\documentclass[]{article}
\usepackage{graphicx}
\usepackage{amsmath}
\usepackage{amssymb}
\usepackage{amsfonts}
\usepackage{fancyhdr}
\usepackage[headheight=65pt,tmargin=150pt,headsep=95pt]{geometry}
\usepackage{ragged2e}

\pagestyle{myheadings}
\markright{Extra Solar Lab Report\hfill Lewis McNish\hfill 1/1/2023\hfill}

\title{\textbf{Identifying Extra Solar Planets and their Key Features using the Doppler Wobble and Planetary Transits Methods}}
\author{Lewis M$^{c}$Nish (University of Glasgow)}
\date{1/1/2023}






\begin{document}
\maketitle

\begin{abstract}
This is the abstract

\end{abstract}
\newpage


% All relevant sections for Method 1 (Doppler Wobble)

\twocolumn
\section*{Introduction and Background}

\section*{Aims}
Understand the effect of Doppler shifts on the intensity of stellar spectra and use the Python scipy.optimize library to determine “best-fit” radial velocities from high- resolution spectra observed at different epochs
Derive a radial velocity curve i.e. radial velocity as a function of orbital phase for each star, and use fitting to estimate the amplitude of each curve 
Estimate the mass and semi-major axis of each planet

\section*{Method}

\section*{Results}

\section*{Analysis}
\section*{Discussion}

\section*{Conclusion}

\section*{References}

% All relevant sections for Planetary transits method
\newpage
\section*{Introduction and Background}

\section*{Aims}

Obtain a phase-folded photometric light curve for a star with a transiting planetary companion. Use this to estimate the radius and orbital semi-major axis of the planet
Apply the method of least-squares to estimate mean apparent magnitudes during the transit and non-transit phase. Hence estimate the radius of the planet 





\section*{Method}

\section*{Results}

\section*{Analysis}
\section*{Discussion}

\section*{Conclusion}

\section*{References}
\end{document}